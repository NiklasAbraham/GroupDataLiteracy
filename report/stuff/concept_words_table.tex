\documentclass[a4paper,10pt]{article}
\usepackage[margin=1in]{geometry}
\usepackage{booktabs}

\begin{document}

\section*{Comparison of Dense and Sparse Concept Extraction}

\begin{table}[h]
\centering
\small
\begin{tabular}{clrclr}
\toprule
\textbf{Rank} & \textbf{Dense Concept} & \textbf{Dense Score} & \textbf{Sparse Concept} & \textbf{Sparse Score} \\
\midrule
1 & arguing & 0.544049 & cinema & 0.462526 \\
2 & filming & 0.531092 & sectional & 0.327910 \\
3 & inside & 0.525644 & tourist & 0.403241 \\
4 & socializing & 0.523859 & return & 0.457992 \\
5 & reliving & 0.523364 & characteristic & 0.362480 \\
6 & yelling & 0.521545 & female & 0.425856 \\
7 & shouting & 0.521472 & police & 0.416033 \\
8 & pretending & 0.520833 & feelings & 0.471460 \\
9 & communicating & 0.518198 & shop & 0.408851 \\
10 & someone & 0.516860 & friend & 0.441347 \\
11 & backstage & 0.516742 & slut & 0.410002 \\
12 & happening & 0.515222 & humor & 0.433643 \\
13 & humans & 0.513191 & hombre & 0.442704 \\
14 & halftime & 0.513142 & search & 0.410907 \\
15 & altercation & 0.512606 & terrorist & 0.422032 \\
16 & chattering & 0.511847 & tranquillity & 0.469650 \\
17 & talks & 0.511800 & routine & 0.414639 \\
18 & sweating & 0.511625 & discourse & 0.422368 \\
19 & roleplaying & 0.509357 & life & 0.427994 \\
20 & sneezing & 0.509242 & television & 0.454755 \\
21 & sit-in & 0.508355 & scenery & 0.469170 \\
22 & check-in & 0.507920 & light & 0.465169 \\
23 & daydreaming & 0.507528 & period & 0.374559 \\
24 & hurrying & 0.506722 & parent & 0.362549 \\
25 & somebody & 0.506603 & election & 0.382254 \\
26 & pleading & 0.506445 & theatre & 0.446549 \\
27 & anecdote & 0.506024 & relation & 0.410553 \\
28 & sit-down & 0.505426 & fashion & 0.347873 \\
29 & haggling & 0.504629 & boat & 0.406669 \\
30 & playback & 0.504442 & image & 0.388053 \\
\bottomrule
\end{tabular}
\caption{Comparison of top 30 concepts extracted using dense and sparse embedding methods.}
\label{tab:concept_comparison}
\end{table}

\vspace{1em}

\textbf{Movie Plot:} The film is set in the tense period in the Israel-Palestinian peace process shortly after the assassination of Yitzhak Rabin and the election of Benjamin Netanyahu, with the strained relations implied but not explicitly depicted. It is divided into two major sections, all loosely tied together as the story of Suleiman's return to the West Bank and Israel. The character of Suleiman in the film is described only as `E.S.' E.S. returns from a twelve-year exile in New York City and is now in unfamiliar territory. Within the film, no real plot or character development emerges. A series of mostly unconnected scenes take place one after the other in documentary film like fashion. The gradual accumulation of images and dialogue start without conclusion presenting an unsettling kind of feeling, which was meant to convey the quality of life led by Palestinians given their statelessness. The first, and lightest, section is the ``Nazareth Personal Diary'', featuring warm observations of his family and his relatives' lives. Some of the notable vignettes include the dull yet comedic routines of the proprietor of a souvenir shop called ``the Holyland'' in which he fills bottles of alleged holy water from his own tap and fails to keep a cheap camel statuette from falling over on his shelves. E.S. and the shop owner spend time sitting in front of the stop waiting futilely for tourists to stop by. A boat full of Arab men fish, as one of the men bashes various Palestinian families that his friend does not belong to while praising the one that his friend does belong to. Suleiman also interviews a Russian Orthodox cleric who rails against the tourists polluting the Sea of Galilee. A short middle segment shows E.S. getting up to speak at a conference on Palestinian film making; the microphone immediately begins feeding back and he leaves the podium. The last section, ``Jerusalem Political Diary, has a quicker narrative pace and a more overtly ideological message. Absurd humor is evoked alongside feelings of paranoia in the characters felt by the broader society; for example, what first appears to be a terrorist's hand grenade held by a Palestinian turns out to be a cigarette lighter. Suleiman discovers an Israeli policeman's walkie-talkie, and he then meets up with a single young Arab woman who is engaging in a search for an apartment that is just as fruitless as the two men's search for tourists. The woman, who speaks fluent Hebrew, is told by Jewish landlords that they do not rent to Arabs, while an Arab landlord tells her to live at home in accordance to Islamic tradition. She uses the walkie-talkie to play various pranks on the Israeli police, at one point singing an overly malevolent version of Israel's national anthem over the air. In the last part of the film, the woman stages a piece in which the police unwittingly participate as a member of a guerrilla theatre group. The end comes after a long shot of Suliman's parents sleeping, with all the lights off and Israeli material playing on their television.

\newpage

\section*{Comparison of Dense and Sparse Concept Extraction (Movie 2)}

\begin{table}[h]
\centering
\small
\begin{tabular}{clrclr}
\toprule
\textbf{Rank} & \textbf{Dense Concept} & \textbf{Dense Score} & \textbf{Sparse Concept} & \textbf{Sparse Score} \\
\midrule
1 & mitzvah & 0.526642 & marriage & 0.519790 \\
2 & marriage & 0.519790 & engagement & 0.379622 \\
3 & yeshiva & 0.505100 & widow & 0.456806 \\
4 & filming & 0.494214 & delay & 0.414525 \\
5 & jeremiah & 0.486933 & husband & 0.430518 \\
6 & bride & 0.484931 & times & 0.355075 \\
7 & pleading & 0.484276 & death & 0.414765 \\
8 & torah & 0.483665 & father & 0.353841 \\
9 & wedding & 0.482318 & proposal & 0.388070 \\
10 & shariah & 0.481167 & family & 0.420231 \\
11 & shia & 0.479988 & mother & 0.410383 \\
12 & married & 0.479422 & cinema & 0.432136 \\
13 & happening & 0.478301 & wedding & 0.482318 \\
14 & sharia & 0.477847 & girl & 0.437378 \\
15 & begging & 0.476959 & house & 0.399123 \\
16 & crying & 0.475215 & tragedy & 0.443515 \\
17 & jeopardy & 0.475045 & dream & 0.347910 \\
18 & bridesmaid & 0.473968 & sister & 0.447530 \\
19 & birthing & 0.472761 & baby & 0.414906 \\
20 & jew & 0.471533 & scenario & 0.406843 \\
21 & wishing & 0.471087 & offer & 0.400715 \\
22 & yemeni & 0.470966 & friend & 0.411713 \\
23 & jealousy & 0.470875 & hombre & 0.382067 \\
24 & someone & 0.469350 & life & 0.391976 \\
25 & resolving & 0.469068 & son & 0.328957 \\
26 & pregnancy & 0.468715 & year & 0.353359 \\
27 & granddaughter & 0.468101 & possibility & 0.447583 \\
28 & cheating & 0.467979 & prospect & 0.394781 \\
29 & midwife & 0.467928 & country & 0.341517 \\
30 & waiting & 0.467740 & idea & 0.388355 \\
\bottomrule
\end{tabular}
\caption{Comparison of top 30 concepts extracted using dense and sparse embedding methods.}
\label{tab:concept_comparison_2}
\end{table}

\vspace{1em}

\textbf{Movie Plot:} Shira Mendelman, an 18-year-old Hasidic girl living in Tel Aviv, is looking forward to an arranged marriage with a young man whom she likes. However, on Purim, her family suffers a tragedy when Shira's older sister Esther dies in childbirth. Shira's father subsequently delays the engagement so as not to have to deal with an empty house so soon after Esther's death. Esther's husband, Yochay, begins to regularly bring their son, Mordechai, to the Mendelman's house, where Shira cares for him. One day, Yochay's mother approaches Shira's mother, Rivka, about the possibility of Yochay remarrying, believing it to be best for Mordechai. She plans to suggest an offer from a widow in Belgium. Rivka is distraught by the idea of Mordechai being taken out of the country, and suggests that Yochay marry Shira instead. He and Shira both initially oppose the prospect, though he eventually warms to it, and she agrees to take it into consideration on learning that her previous engagement has been called off due to her father's delays. However, she had hoped for a younger husband; her dream was of someone who would discover married life for the first time together with her. Shortly afterwards, Frieda, a friend of Esther who has never received any marriage proposals, tells Shira that Esther would have preferred that Yochay marry her in the event of her death. As a result, Shira tells Yochay that Frieda is more suitable, which he takes as an affront. Shira and Yochay remain distant from one another afterwards, and he announces that he plans to move with Mordechai to marry the widow in Belgium. Shira, pressured by her family, agrees to go forward with the engagement to Yochay, believing it to be the best scenario for everyone. However, the rabbi realizes that Shira is half-hearted, and he refuses to condone the marriage. Time passes, and Shira eventually concludes that she was meant to be with Yochay and his baby. She approaches the rabbi and asks again that she and Yochay be married, and he agrees this time. The film closes with their wedding.

\end{document}
