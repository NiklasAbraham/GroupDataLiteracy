% introduction to what we expect

As movie genres provide a meaningful taxonomy with potential temporal evolution patterns, we examine semantic drift across different time periods. To this end, embeddings are first grouped by genre $g$ into discrete time periods $\tau$, forming the set $\mathcal{M}_g^{(\tau)}$ of plot embeddings. For each group, two alternative representative embeddings are computed: the \textbf{centroid} (arithmetic mean) $\bar{\mathbf{e}}_g^{(\tau)}$ and the \textbf{medoid} (cosine distance minimizer embedding) $\tilde{\mathbf{e}}_g^{(\tau)}$.

With an arbitrary number of years $\Delta t$ per group, the period index $\tau$ is calculated by flooring the movie year to the nearest multiple of $\Delta t$: 

\begin{equation}
    \tau = \left\lfloor \frac{\text{year}}{\Delta t} \right\rfloor \cdot \Delta t
    \label{eq:tau}
\end{equation}

We computed the following metrics to analyse the drift dynamics across the groups:

\textbf{Genre drift and acceleration: }drift (Equation \ref{eq:drift}) measures displacement between representative embeddings of consecutive periods, capturing how a genre's semantic center evolves over time. Acceleration quantifies the change in drift between consecutive periods.

\begin{equation}
    % === 3. Drift Velocity ===
    \mathbf{d}_g^{(\tau)} = \bar{\mathbf{e}}_g^{(\tau + \Delta t)} - \bar{\mathbf{e}}_g^{(\tau)}
    \label{eq:drift}
\end{equation}
\textbf{Inter-genre distance: }determines cosine distance between representatives of each pair of genres for each year, enabling pairwise comparison between specific genres.

Due to group size differences between time periods, two alternative normalization 
approaches have been employed: (1) downsampling, ensuring equal sampling error 
across groups, and (2) z-score normalization, which accounts for the standard 
error of the difference between group means:
\begin{equation}
   \hat{v}_g^{(\tau)} = \frac{v_g^{(\tau)}}{\sigma_{\text{pooled}} \cdot \sqrt{\frac{1}{n_g^{(\tau)}} + \frac{1}{n_g^{(\tau + \Delta t)}}}}
\end{equation}
where $\sigma_{\text{pooled}}$ is the pooled within-group standard deviation of cosine distances, and $n_g^{(\tau)}$ is the number of movies in genre $g$ at time $\tau$.

None of the genre-based analyses yielded statistically significant results, suggesting that genres may be too broad as analytical categories and any underlying patterns are likely obscured by noise. Figure \ref{fig:drift} illustrates temporal drift for the three most popular genres, computed over 5-year periods. Each period was downsampled to 145 movies, with 95\% confidence intervals estimated via bootstrapping (1,000 samples).

\begin{figure}[htbp]
    \centering
    \includegraphics[width=0.6\textwidth]{figures/genre_drift.png}
    \caption{Genre drift from 1930 to 2025 of three most popular genres over 5-year periods}
    \label{fig:drift}
\end{figure}